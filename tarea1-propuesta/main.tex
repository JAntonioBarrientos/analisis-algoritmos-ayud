\documentclass[12pt,letterpaper]{article}
\usepackage[utf8]{inputenc}
\usepackage[spanish]{babel}
\usepackage{amsmath}
\usepackage{amsfonts}
\usepackage{amssymb}
\usepackage{graphicx}
\usepackage{geometry}
\usepackage{listings}
\usepackage{xcolor}

\geometry{left=2.5cm, right=2.5cm, top=2.5cm, bottom=2.5cm}

% --- Configuración para listados de código ---
\definecolor{codegreen}{rgb}{0,0.6,0}
\definecolor{codegray}{rgb}{0.5,0.5,0.5}
\definecolor{codepurple}{rgb}{0.58,0,0.82}
\definecolor{backcolour}{rgb}{0.95,0.95,0.92}

\lstdefinestyle{mystyle}{
    backgroundcolor=\color{backcolour},
    commentstyle=\color{codegreen},
    keywordstyle=\color{magenta},
    numberstyle=\tiny\color{codegray},
    stringstyle=\color{codepurple},
    basicstyle=\footnotesize\ttfamily,
    breakatwhitespace=false,
    breaklines=true,
    captionpos=b,
    keepspaces=true,
    numbers=left,
    numbersep=5pt,
    showspaces=false,
    showstringspaces=false,
    showtabs=false,
    tabsize=2
}
\lstset{style=mystyle}

% --- Encabezado ---
\title{Tarea 1: Propuesta de ejercicios}
\author{Jose Antonio Barrientos Sanchez}
\date{\today}

\begin{document}
\maketitle

\hrulefill
\vspace{1cm}

\subsection*{Ejercicio 1}

Ordene la siguiente lista de funciones de menor a mayor según su tasa de crecimiento asintótico. Es decir, si una función $f(n)$ precede a $g(n)$, entonces $f(n) = O(g(n))$.

$$
\begin{array}{ccccc}
    \sqrt{n} & n \log_2(n) & 2^n & n^2 & \log_2(n) \\
    n! & n & 1000 & n^3 & \log_2(\log_2(n)) \\
\end{array}
$$

\subsection*{Ejercicio 2}
Determine la complejidad temporal del siguiente fragmento de código.

\begin{lstlisting}[language=Python, caption=Bucle simple]
def funcion_ejemplo(n):
    suma = 0
    i = 1
    while i <= n:
        suma += i
        i = i * 2
    return suma
\end{lstlisting}



\subsection*{Ejercicio 3}
Usando la definición formal de Big O, demuestre que $5n^3+ 3n + 2$ es $O(n^3)$.

\subsection*{Ejercicio 4}

Demuestre que para todo $k \geq 0$ y todo conjunto de constantes reales $\{a_k, a_{k-1}, \dots, a_1, a_0\}$, se cumple que:
    $$ a_k n^k + a_{k-1} n^{k-1} + \dots + a_1 n + a_0 = O(n^k) $$

\subsection*{Ejercicio 5}
    
    Muestre que para cualesquiera constantes reales $a$ y $b$, con $b > 0$, se cumple que:
    $$ (n + a)^b = \Theta(n^b) $$



\end{document}