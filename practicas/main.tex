\documentclass{article}
\usepackage[utf8]{inputenc}
\usepackage[spanish]{babel}
\usepackage{geometry}
\usepackage{hyperref}

\geometry{a4paper, margin=1in}
\hypersetup{
    colorlinks=true,
    linkcolor=blue,
    filecolor=magenta,      
    urlcolor=cyan,
}

\title{\textbf{Práctica 2: Recursión e Introducción a Programación Dinámica}}
\author{Profesor: Oscar Hernández Constantino\\ Ayudantes: Ilse Gema Peña Bernardino \\ José Antonio Barrientos Sánchez}
\date{\today}

\begin{document}
\maketitle


\section*{Objetivo}
Resolver una serie de ejercicios introductorios y de programación dinámica (DP) para fortalecer las habilidades de diseño de algoritmos y su correcta implementación.

\section{Instrucciones Generales}

La práctica consiste en resolver los problemas propuestos en la plataforma VJudge. Estos se dividen en ejercicios introductorios y ejercicios de programación dinámica.

\subsection{Plataforma de Trabajo}
Todos los ejercicios deben ser resueltos y enviados a través del concurso \textbf{Práctica 2}, disponible en nuestro grupo de VJudge. 

\subsection{Lenguaje de Programación}
Pueden utilizar \textbf{cualquier lenguaje de programación} que prefieran, siempre y cuando esté disponible en el juez en línea de VJudge. La plataforma es compatible con la mayoría de los lenguajes de programación competitiva (C++, Java, Python, etc.).

\subsection{Modalidad de Trabajo}
El trabajo es \textbf{individual}. Si se detecta plagio o códigos idénticos entre compañeros resultará en una calificación de \textbf{cero (0)} para todos los involucrados, sin excepción.

\section{Criterios de Evaluación}
Para que un ejercicio sea considerado como resuelto, debe recibir el veredicto \textbf{"Accepted" (Aceptado)} por parte del juez en línea. Este veredicto es la única garantía de que la solución no solo es correcta, sino que también cumple con los requisitos de complejidad y eficiencia esperados en tiempo y espacio.

\section{Entregables}
Además de la resolución en VJudge, se deberá entregar un reporte en formato \textbf{PDF} en la tarea correspondiente de Google Classroom.

\subsection{Contenido del Reporte}
El reporte es de \textbf{formato libre} y debe incluir, por cada problema resuelto, lo siguiente:
\begin{enumerate}
    \item \textbf{Análisis de la solución:} Una explicación clara de cómo llegaste al algoritmo final. Detalla tu razonamiento y la lógica detrás del enfoque que elegiste.
    \item \textbf{Solución y complejidad:} Describe el algoritmo final y analiza su complejidad temporal y espacial.
    \item \textbf{Retos de implementación:} Documenta cualquier problema o desafío que enfrentaste al traducir tu algoritmo a código, y cómo lo solucionaste.
\end{enumerate}
\textbf{Importante:} El documento PDF debe incluir su nombre completo y número de cuenta en la primera página.

\section{Dudas y Aclaraciones}
Si tienen cualquier duda sobre la implementación de algún problema o necesitan alguna aclaración, no duden en contactarme.

\bigskip 
\begin{center}
    \textbf{¡Mucho éxito!}
\end{center}

\end{document}